\documentclass{article}
\usepackage[utf8]{inputenc}
\usepackage[T1]{fontenc}
\usepackage{amsmath}      % Für mathematische Symbole und Formeln
\usepackage{listings}     % Für Quellcode-Formatierung
\usepackage{xcolor}       % Für Farben im Quellcode

\title{PacSpedd Documentation}
\author{PacSpedd}
\date{\today}

\begin{document}

\maketitle

\tableofcontents

\section{Get Started}
This is the Instruction for PacSpedd, a Build and Package System, written in Python.

\section{Modules}
PacSpedd has many specially configured Python libraries:

\begin{enumerate}
    \item \textbf{psbase}: Written in Rust, needed for few other modules. Termux gets a special version.
    \item \textbf{psmake}: Make Tool, needed for building packages via Makefile.psmake.
    \item \textbf{psbuild}: PacSpedd Build Library, essential for building PacSpedd packages.
    \item \textbf{psmanager}: PacSpedd Manager, helps search and install PacSpedd packages, both from repositories and locally.
    \item \textbf{pshelper}: PacSpedd Help tool, assists in creating, building, and testing new PacSpedd packages.
\end{enumerate}

\subsection{Usage: pshelper}

\begin{verbatim}
    Usage: pshelper [command] [name]
    Commands:
        new         ==> Create new PacSpedd with new directory
        init        ==> Create new PacSpedd in existing directory
        build       ==> Build new PacSpedd, must be in the root folder of the project
        test        ==> Test the PacSpedd before building it
        depends     ==> Add dependencies to manifest file
        convert     ==> Generate binary from PacSpedd package, must be declared in the manifest file
\end{verbatim}

\subsection{Usage: psmake}

\begin{verbatim}
    Usage: psmake [command] [step]
    Commands:
        make        ==> Need step after this, runs the named step
        generate    ==> Create Makefile from the Makefile.psmake
        run-all     ==> Run all steps in one tour
\end{verbatim}

\subsection{psbase}

\begin{verbatim}
    from psbase import print
    from psbase import cmd
    from psbase import env
    from psbase import PacSpeddBase
\end{verbatim}

\end{document}
